%%%%%%%%%%%%%%%%%%%%%%%%%%%%%%%%%%%%%%%%%
% Beamer Presentation
% LaTeX Template
% Version 1.0 (10/11/12)
%
% This template has been downloaded from:
% http://www.LaTeXTemplates.com
%
% License:
% CC BY-NC-SA 3.0 (http://creativecommons.org/licenses/by-nc-sa/3.0/)
%
%%%%%%%%%%%%%%%%%%%%%%%%%%%%%%%%%%%%%%%%%

%----------------------------------------------------------------------------------------
%	PACKAGES AND THEMES
%----------------------------------------------------------------------------------------

\documentclass{beamer}

\mode<presentation> {

% The Beamer class comes with a number of default slide themes
% which change the colors and layouts of slides. Below this is a list
% of all the themes, uncomment each in turn to see what they look like.

%\usetheme{default}
%\usetheme{AnnArbor}
%\usetheme{Antibes}
%\usetheme{Bergen}
%\usetheme{Berkeley}
%\usetheme{Berlin}
%\usetheme{Boadilla}
%\usetheme{CambridgeUS}
%\usetheme{Copenhagen}
%\usetheme{Darmstadt}
%\usetheme{Dresden}
%\usetheme{Frankfurt}
%\usetheme{Goettingen}
%\usetheme{Hannover}
%\usetheme{Ilmenau}
%\usetheme{JuanLesPins}
%\usetheme{Luebeck}
\usetheme{Madrid}
%\usetheme{Malmoe}
%\usetheme{Marburg}
%\usetheme{Montpellier}
%\usetheme{PaloAlto}
%\usetheme{Pittsburgh}
%\usetheme{Rochester}
%\usetheme{Singapore}
%\usetheme{Szeged}
%\usetheme{Warsaw}

% As well as themes, the Beamer class has a number of color themes
% for any slide theme. Uncomment each of these in turn to see how it
% changes the colors of your current slide theme.

%\usecolortheme{albatross}
%\usecolortheme{beaver}
%\usecolortheme{beetle}
%\usecolortheme{crane}
%\usecolortheme{dolphin}
%\usecolortheme{dove}
%\usecolortheme{fly}
%\usecolortheme{lily}
%\usecolortheme{orchid}
%\usecolortheme{rose}
%\usecolortheme{seagull}
%\usecolortheme{seahorse}
%\usecolortheme{whale}
%\usecolortheme{wolverine}

%\setbeamertemplate{footline} % To remove the footer line in all slides uncomment this line
%\setbeamertemplate{footline}[page number] % To replace the footer line in all slides with a simple slide count uncomment this line

%\setbeamertemplate{navigation symbols}{} % To remove the navigation symbols from the bottom of all slides uncomment this line
}

\usepackage{graphicx} % Allows including images
\usepackage{booktabs} % Allows the use of \toprule, \midrule and \bottomrule in tables
\usepackage{amsmath}
\usepackage{verbatim}


%----------------------------------------------------------------------------------------
%	TITLE PAGE
%----------------------------------------------------------------------------------------

\title[Spectral learning with structure]{Spectral learning for structured partially observable environments} % The short title appears at the bottom of every slide, the full title is only on the title page

\author{Lucas Langer} % Your name
\institute[McGill University] % Your institution as it will appear on the bottom of every slide, may be shorthand to save space
{
McGill University \\ % Your institution for the title page
\medskip
\textit{lucas.langer@mail.mcgill.ca} % Your email address
}
\date{\today} % Date, can be changed to a custom date

\begin{document}

\begin{frame}
\titlepage % Print the title page as the first slide
\end{frame}

\begin{frame}
\frametitle{Overview} % Table of contents slide, comment this block out to remove it
\tableofcontents % Throughout your presentation, if you choose to use \section{} and \subsection{} commands, these will automatically be printed on this slide as an overview of your presentation
\end{frame}

%----------------------------------------------------------------------------------------
%	PRESENTATION SLIDES
%----------------------------------------------------------------------------------------

%------------------------------------------------
\section{Predictive state representations} % Sections can be created in order to organize your presentation into discrete blocks, all sections and subsections are automatically printed in the table of contents as an overview of the talk
%------------------------------------------------

%\subsection{Subsection Example} % A subsection can be created just before a set of slides with a common theme to further break down your presentation into chunks


%------------------------------------------------

\begin{frame}
\frametitle{Structure partially observable environments}
\begin{columns}[c] % The "c" option specifies centered vertical alignment while the "t" option is used for top vertical alignment

\column{.45\textwidth} % Left column and width
\textbf{Structured Environments}
\begin{enumerate}
\item Goal: Predictions
\item Plan: Exploit structure
\item Example: Pacman

\end{enumerate}

\column{.5\textwidth} % Right column and width
\begin{figure}
\includegraphics[width=1.0\linewidth]{lucasplots/pac-man.jpg}
\end{figure}

\end{columns}
\end{frame}

%------------------------------------------------

\begin{frame}
\frametitle{PSRs: The Timing Case}

\begin{itemize}
\item Model environment with a Predictive state representation
\item PSR defined by: $<\alpha_0, \{A_\sigma\},\alpha_\infty\}>$
\item[] $\alpha_0$: Initial weighting on states $1xn$
\item[] $A_\sigma$ Transition matrix $nxn$
\item[] $\alpha_\infty$: Normalizer $nx1$

\item PSRs compute probabilities of observations
\item[] Notation: $\sigma$: one time unit, $\sigma^k$: k time units
\item[] $f(\sigma^k) = \alpha_0*A_\sigma^k*\alpha_\infty$
\item Examples of a PSRs: HMMs, POMDPs
 
\end{itemize}

\end{frame}

%------------------------------------------------

\begin{frame}
\frametitle{Overview of Learning}

\begin{itemize}
\item Spectral algorithms can learn PSRs 
\item Flavour of learning algorithm: 
\item[] Step 1: Represent data as a matrix
\item[] Step 2: Singular value decomposition
\item[] Step 3: Pick number of states for PSR
\item[] Step 4: Learn the PSR with matrix computations
\end{itemize}

\end{frame}

\section{The Base System}
%------------------------------------------------

\begin{frame}
\frametitle{The Base System}
\begin{itemize}

\item Number representations: $11 = 2^3 + 2^1 + 2^0$

\item Timing queries $f(a^11) = \alpha*A_(a^8)*A_(a^2)*A_(a^1)$

\item Motivation: 
\item[] 1) Express transitions directly to avoid error build up
\item[] 2) Faster queries. Discussion $\alpha_0*(A_\sigma)^k$
\end{itemize}
\end{frame}

%------------------------------------------------
\begin{comment}
\begin{frame}
\frametitle{The Base System Cont.}
\begin{itemize}

\item When taking a reduced model compounding errors are a threat

\item Analogy to rounding:  Round(51.63*34.12) v.s Round(51.63) * Round(34.12)

\item Let $\pi$: n states --> k states be the projection operator from a system with n states to the k-best states

\item
$f_Base(\sigma^128) = (\pi*\alpha_0)*(\pi*A_\sigma^128)*(\pi*\alpha_\infty)$

$f_Naive(x) = (\pi*\alpha_0)*(\pi*A_\sigma^128)*(\pi*\alpha_\infty)$ 

\end{itemize}

\end{frame}
\end{comment}

%------------------------------------------------


\section{Experimental results}
%------------------------------------------------

\begin{frame}
\frametitle{Timing with the Base}
Agent goes through loops until leaving through an exit state. Exit states have transition probabilities of 0.4 and 0.6. Loop lengths are 64 and 16.
\begin{figure}
\includegraphics[width=0.8\linewidth]{lucasplots/monImages/doubleLoopImage.png}
\end{figure}
\end{frame}

%------------------------------------------------


\begin{frame}
\frametitle{Double Loop Results}

$||f_A - f_ABar|| = (\sum (f_A(x) - f_ABar)^2)^(0.5)$ 

\begin{figure}
\includegraphics[width=0.5\linewidth]{lucasplots/monImages/DoubleLoopTiming0.png}
\includegraphics[width=0.5\linewidth]{lucasplots/monImages/doubleLoopImage.png}

\end{figure}
\end{frame}

%------------------------------------------------

\begin{frame}
\frametitle{PacMan Labyrinth Results}

$||f_A - f_ABar|| = (\sum (f_A(x) - f_ABar)^2)^(0.5)$ 

\begin{figure}
\includegraphics[width=0.8\linewidth]{lucasplots/monImages/PacmanLabyrinth.png}
\end{figure}
\end{frame}

%------------------------------------------------

\begin{frame}
\frametitle{Distance Predictions}
We use $\alpha_0*A_\sigma^k$ as a representation of state. Linear regression gives us a distance weighting on states.
\begin{figure}
\includegraphics[width=0.8\linewidth]{lucasplots/monImages/Distance_Predictions.png}
\end{figure}
\end{frame}

\begin{frame}
\frametitle{Wall Color Predictions}
We paint the first loop green and the second loop red.
\begin{figure}
\includegraphics[width=0.5\linewidth]{lucasplots/monImages/WallColorPredictions.png}
\includegraphics[width=0.5\linewidth]{lucasplots/monImages/WallColorPredictions.png}


\end{figure}
\end{frame}

%------------------------------------------------

%------------------------------------------------

\section{Learning the Base System}
%------------------------------------------------

\begin{frame}
\frametitle{Picking the Base System}
\begin{itemize}


\item How do we pick transition operators?
\item[] Observations: \{$a^{30}$:10, $a^{60}$:5, $b^{18}$:15\}

\item[] Desired Base System: {$A_a^{30}$, $A_b^{18}$, $A_a$, $A_b$}

\item Substring properties: 
\textbf{long}, \textbf{frequent}, \textbf{diverse}

\item Solution: iterative greedy heuristic 


\end{itemize}
\end{frame}

%------------------------------------------------

\begin{frame}
\frametitle{Computing with the Base System}
\begin{itemize}


\item Using the Base System well involves requires good \textbf{string partitions}

\item[] Query string: "abcacb", Base System = $\{A_{ab}, A_{bca}, A_{cb}, A_a, A_b \}$ 


\item[] Desired partition: "a|bca|cb""

\item Goal: minimize matrices used

\item Solution: dynamic programming

\end{itemize}
\end{frame}

%------------------------------------------------

\begin{frame}
\frametitle{Conclusion and Future Work}
\begin{itemize}

\item What's left for the Base System?
\item[] 1) Theoretical analysis

\item[] 2) Test heuristics for selection and querying

\item[] 3) Further optimize heuristics


\end{itemize}
\end{frame}

%------------------------------------------------

\begin{frame}
\Huge{\centerline{Questions? Comments?}}
\end{frame}

%					DONE

%------------------------------------------------
%------------------------------------------------
%------------------------------------------------
%------------------------------------------------
%-----------------------------------------------
%------------------------------------------------
%------------------------------------------------

%					DONE


%----------------------------------------------------------------------------------------

\end{document} 