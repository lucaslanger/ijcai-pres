%%%%%%%%%%%%%%%%%%%%%%%%%%%%%%%%%%%%%%%%%
% Beamer Presentation
% LaTeX Template
% Version 1.0 (10/11/12)
%
% This template has been downloaded from:
% http://www.LaTeXTemplates.com
%
% License:
% CC BY-NC-SA 3.0 (http://creativecommons.org/licenses/by-nc-sa/3.0/)
%
%%%%%%%%%%%%%%%%%%%%%%%%%%%%%%%%%%%%%%%%%

%----------------------------------------------------------------------------------------
%	PACKAGES AND THEMES
%----------------------------------------------------------------------------------------

\documentclass{beamer}

\mode<presentation> {

% The Beamer class comes with a number of default slide themes
% which change the colors and layouts of slides. Below this is a list
% of all the themes, uncomment each in turn to see what they look like.

%\usetheme{default}
%\usetheme{AnnArbor}
%\usetheme{Antibes}
%\usetheme{Bergen}
%\usetheme{Berkeley}
%\usetheme{Berlin}
%\usetheme{Boadilla}
%\usetheme{CambridgeUS}
%\usetheme{Copenhagen}
%\usetheme{Darmstadt}
%\usetheme{Dresden}
%\usetheme{Frankfurt}
%\usetheme{Goettingen}
%\usetheme{Hannover}
%\usetheme{Ilmenau}
%\usetheme{JuanLesPins}
%\usetheme{Luebeck}
\usetheme{Madrid}
%\usetheme{Malmoe}
%\usetheme{Marburg}
%\usetheme{Montpellier}
%\usetheme{PaloAlto}
%\usetheme{Pittsburgh}
%\usetheme{Rochester}
%\usetheme{Singapore}
%\usetheme{Szeged}
%\usetheme{Warsaw}

% As well as themes, the Beamer class has a number of color themes
% for any slide theme. Uncomment each of these in turn to see how it
% changes the colors of your current slide theme.

%\usecolortheme{albatross}
%\usecolortheme{beaver}
%\usecolortheme{beetle}
%\usecolortheme{crane}
%\usecolortheme{dolphin}
%\usecolortheme{dove}
%\usecolortheme{fly}
%\usecolortheme{lily}
%\usecolortheme{orchid}
%\usecolortheme{rose}
%\usecolortheme{seagull}
%\usecolortheme{seahorse}
%\usecolortheme{whale}
%\usecolortheme{wolverine}

%\setbeamertemplate{footline} % To remove the footer line in all slides uncomment this line
%\setbeamertemplate{footline}[page number] % To replace the footer line in all slides with a simple slide count uncomment this line

%\setbeamertemplate{navigation symbols}{} % To remove the navigation symbols from the bottom of all slides uncomment this line
}

\usepackage{graphicx} % Allows including images
\usepackage{booktabs} % Allows the use of \toprule, \midrule and \bottomrule in tables
\usepackage{amsmath}
\usepackage{verbatim}
\usepackage{tikz}
\usepackage{subcaption}
\usepackage{subfig}
\usepackage{caption}
\usepackage{pifont}
\usetikzlibrary{shapes.geometric, arrows}


%----------------------------------------------------------------------------------------
%	TITLE PAGE
%----------------------------------------------------------------------------------------

\title[Multi-Step PSRs]{Learning Multi-Step Predictive State Representations} % The short title appears at the bottom of every slide, the full title is only on the title page

\author{Lucas Langer (\ding{168}), Borja Balle (\ding{70}), Doina Precup (\ding{168}) \\ McGill University (\ding{168}), Lancaster University (\ding{70})} % Your name
\institute[McGill University] % Your institution as it will appear on the bottom of every slide, may be shorthand to save space
{
%McGill University \\ % Your institution for the title page
\\
\medskip
%\textit{lucas.langer@mail.mcgill.ca} % Your email address
}
\date{\today} % Date, can be changed to a custom date

\begin{document}


\begin{frame}
\titlepage % Print the title page as the first slide
\end{frame}

\begin{comment}
\begin{frame}
\frametitle{Overview} % Table of contents slide, comment this block out to remove it
\tableofcontents % Throughout your presentation, if you choose to use \section{} and \subsection{} commands, these will automatically be printed on this slide as an overview of your presentation
\end{frame}
\end{comment}


%----------------------------------------------------------------------------------------
%	PRESENTATION SLIDES
%----------------------------------------------------------------------------------------

%------------------------------------------------
\section{Predictive state representation (PSR)} % Sections can be created in order to organize your presentation into discrete blocks, all sections and subsections are automatically printed in the table of contents as an overview of the talk
%------------------------------------------------

%\subsection{Subsection Example} % A subsection can be created just before a set of slides with a common theme to further break down your presentation into chunks


%------------------------------------------------

\begin{frame}
\frametitle{Predictive State Representations (PSR)}
\begin{columns}[c] % The "c" option specifies centered vertical alignment while the "t" option is used for top vertical alignment

\column{.95\textwidth} % Left column and width
\textbf{Motivation}
\begin{enumerate}
\item Make predictions in partially observable environments 
\item Learn a representation of hidden states
\item[]
\end{enumerate}

\textbf{Advantages over HMMs}
\begin{enumerate}
\item Statistically consistent and well understood learning efficiency
\item Smaller representations of state
\item[] [Boots et al., 2011; Hsu et al., 2009; Bailly et al., 2010]
\item[]

\end{enumerate}

\textbf{Why Multi-Step PSRs?}
\begin{enumerate}
\item Leverage structure of underlying environment
\item Directly apply transitions which occur at longer time-scales 

\end{enumerate}

\end{columns}
\end{frame}

%------------------------------------------------

\begin{frame}
\frametitle{PSR: The single observation case}

\begin{enumerate}
\item PSR defined by: $\langle \alpha_0, \{A_\sigma\},\alpha_\infty \rangle$, where
\item[] $\alpha_0$ is an initial weighting on states $(1xn)$
\item[] $A_\sigma$ is a transition matrix $(nxn)$
\item[] $\alpha_\infty$ is a normalizer $(nx1)$
\item[] n: number of latent states
\item[]
\item PSRs compute probabilities of observations
\item[] $f(\sigma^k) = \alpha_0 \cdot A_\sigma^k \cdot \alpha_\infty$
\item[]
\item[] [Littman et al., 2001; Singh et al., 2004; Rosencrantz et al., 2004]

\end{enumerate}

\end{frame}

%------------------------------------------------

\begin{frame}
\frametitle{Spectral Learning of PSRs}

\begin{enumerate}
\item Represent data as a Hankel Matrix
\item Singular Value Decomposition
\item Pick model size
\item Linear Algebra
\item Result: $\langle \alpha_0, \{A_\sigma\},\alpha_\infty \rangle$ 
\item[]
\item[] [Boots et al., 2011]
\end{enumerate}

\end{frame}

\section{The Base System: extending PSRs}
%------------------------------------------------

\begin{frame}
\frametitle{Adding multi-step operators: The Base System}
\begin{enumerate}

\item Learn $\{A_{\sigma^2}, A_{\sigma^4}, A_{\sigma^8}, ... A_{\sigma^{2^N}}\}$ as extra transition operators
\item[] Operators learned separately so $A_{\sigma} \cdot A_{\sigma} \neq A_{\sigma^2}$
\item[]

\item $f(\sigma^{11}) = \alpha_0 \cdot A_{\sigma^8} \cdot A_{\sigma^2} \cdot A_{\sigma^1} \cdot \alpha_\infty$
\item[]

\item \textbf{Why might this help?}
\item[] Reduce build up of error
\item[] Faster computations

\end{enumerate}
\end{frame}

%------------------------------------------------
\begin{comment}
\begin{frame}
\frametitle{The Base System Cont.}
\begin{itemize}

\item When taking a reduced model compounding errors are a threat

\item Analogy to rounding:  Round(51.63*34.12) v.s Round(51.63) * Round(34.12)

\item Let $\pi$: n states --> k states be the projection operator from a system with n states to the k-best states

\item
$f_Base(\sigma^128) = (\pi*\alpha_0)*(\pi*A_\sigma^128)*(\pi*\alpha_\infty)$

$f_Naive(x) = (\pi*\alpha_0)*(\pi*A_\sigma^128)*(\pi*\alpha_\infty)$ 

\end{itemize}

\end{frame}
\end{comment}

%------------------------------------------------


\section{Experimental results}
%------------------------------------------------

\begin{comment}
\begin{frame}
\frametitle{Timing with the Base}
Agent drives around loops until leaving through an exit state. 

\begin{figure}
\includegraphics[width=0.8\linewidth]{lucasplots/monImages/doubleLoopImage.png}
\end{figure}
\end{frame}
\end{comment}
%------------------------------------------------


\begin{frame}
\frametitle{Base System Performance for Loops}


%\begin{itemize}
%\item Left Figure: No noise in durations
%\item Right Figure: Noise in loops

%\end{itemize}

%\hspace{1cm} \textbf{64-16 Loop Lengths}
%\hspace{2.5cm} \textbf{47-27 Loop Lengths}

\begin{columns}[c]

\column{.75\textwidth} % Left column and width
\begin{figure}

\includegraphics[width=1.0\linewidth]{uCOREPICS/DoubleLoop64-16.png}
\end{figure}

%\column{.4\textwidth} % Right column and width
%\begin{figure}
%\includegraphics[width=1.0\linewidth]{uCOREPICS/DoubleLoop47-27.png}
%\end{figure}

\column{.25\textwidth}
\begin{figure}
\includegraphics[width=1.0\linewidth]{lucasplots/monImages/doubleLoopImage.png}
\end{figure}

\end{columns}

\begin{itemize}
%\item[] Base System dominates for smaller models

%\item[] $||f - \hat{f}|| = \sqrt{\sum\nolimits_{x \in observations}(f(x) - \hat{f(x)})^2}$ 
\end{itemize}

\end{frame}

%------------------------------------------------


\begin{frame}
\frametitle{Pacman-like Labyrinth}

%\begin{itemize}
%\item Left figure: Timing predictions
%\item Right figure: Distance predictions
%\end{itemize}


%\hspace{1.5cm} Timing Predictions
%\hspace{3cm} Distance Predictions

\begin{columns}[c]

\column{.7\textwidth} % Left column and width
\begin{figure}
\includegraphics[width=1.0\linewidth]{uCOREPICS/PacManTiming.png}
\end{figure}

\begin{itemize}

%\item[] $||f - \hat{f}|| = \sqrt{\sum\nolimits_{x \in observations}(f(x) - \hat{f(x)})^2}$ 
\end{itemize}


\column{.3\textwidth} % Right column and width

\begin{figure}
\centering
\subcaptionbox{Pacman}{%
  \includegraphics[width=0.5\textwidth]{lucasplots/pac-man.jpg}%
  }\par\medskip
\subcaptionbox{Graph\label{cw_25}}{%
  \includegraphics[width=2.0cm,height=3cm]{uCOREPICS/graphPacMan}%
  }\par\medskip        
\end{figure}

\end{columns}


\end{frame}

%------------------------------------------------
\begin{comment}
\begin{frame}
\frametitle{Wall Color Predictions}
We paint the first loop green and the second loop blue
\begin{figure}
\includegraphics[width=0.6\linewidth]{uCOREPICS/MultipleObservations.png}
\includegraphics[width=0.4\linewidth]{lucasplots/monImages/doubleLoopImageMO.png}

\end{figure}

\begin{itemize}

%\item[] $||f - \hat{f}|| = \sqrt{\sum\nolimits_{x \in observations}(f(x) - \hat{f(x)})^2}$ 
\end{enumerate}

\end{frame}
\end{comment}
%------------------------------------------------

%------------------------------------------------

\section{Learning the Base System}
%------------------------------------------------

\begin{frame}
\frametitle{In general, which operators to learn?}
\begin{enumerate}

\item Observations: \{$a^{30}b^{15}$, $a^{60}$, $b^{15}$\}

\item[] Desired operators: $A_{a^{30}b^{15}}$, $A_{a^{60}}$, $A_{a^{30}}$, $A_{b^{15}}$, $A_a$, $A_b$
\item[]

\item Sub-string properties: 
\textbf{long}, \textbf{frequent}, \textbf{diverse}
\item[] Structured environments should be easier
\item[]

\item Trade-off between number of operators and learning time
\item[] Iterative greedy heuristic works well
\item[] Could also try an entropy based approach

\end{enumerate}
\end{frame}

%------------------------------------------------

\begin{frame}
\frametitle{How should we execute queries}
\begin{enumerate}

\item \textbf{Minimize number of matrices}
\item[] Represent transitions as compactly as possible 
\item[]

\item Query string: "abcacb", Operators = $\{A_{ab}, A_{bca}, A_{cb}, A_a, A_b \}$ 


\item[] Desired partition: "a|bca|cb"
\item[] Computation: $f(abcacb)=\alpha_0 \cdot A_a \cdot A_{bca} \cdot A_{cb} \cdot \alpha_\infty$ 
\item[]

\item Solution: dynamic programming
\item[] State update for online applications

\end{enumerate}
\end{frame}

%------------------------------------------------

\begin{frame}
\frametitle{Performance of Heuristics}

\begin{columns}[c]

\column{.4\textwidth} % Left column and width
\begin{figure}

\includegraphics[width=1.0\linewidth]{uCOREPICS/PacManTimingHeuristicsIncluded.png}
\end{figure}

\column{.4\textwidth} % Right column and width
\begin{figure}
\includegraphics[width=1.0\linewidth]{uCOREPICS/MultipleObservationsHeuristics.png}
\end{figure}

\column{.2\textwidth}
\begin{figure}
\includegraphics[width=1.0\linewidth]{lucasplots/monImages/doubleLoopImageMO.png}
\end{figure}

\end{columns}

\end{frame}

%------------------------------------------------
\begin{comment}
\begin{frame}
\frametitle{The Big Picture}

\tikzstyle{startstop} = [rectangle, rounded corners, minimum width=3cm, minimum height=1cm,text centered, draw=black, fill=green!30]

\tikzstyle{io} = [rectangle, rounded corners, minimum width=3cm, minimum height=1cm,text centered, draw=black, fill=green!30]

\tikzstyle{process} = [rectangle, minimum width=3cm, minimum height=1cm, text centered, draw=black, fill=red!30]

\tikzstyle{decision} = [rectangle, minimum width=3cm, minimum height=1cm, text centered, draw=black, fill=green!30]


\tikzstyle{arrow} = [thick,->,>=stealth]


\begin{tikzpicture}[node distance=2cm]

\node (in2) [startstop] {Partially Observable Environment};

\node (in1) [process, below of=in2] {Pick Operators};

\node (dec1) [decision, below of=in1, yshift=-0.5cm] {Learn PSR};

\node (io1) [io, left of=dec1, xshift = -2.5cm] {Query String };

\node (out1) [process, right of=dec1, xshift = 2.5cm] {Partition};
\node (stop) [startstop, below of=out1] {Probability};

\draw [arrow] (in1) -- node[anchor=east] {  \{$A_{a},A_{cb}$,...\} } (dec1);

\draw [arrow] (in2) -- node[anchor=east] {\{abcacb,bca,a,...\}} (in1);

\draw [arrow] (io1) -- node[anchor=south] {abcacb} (dec1);
\draw [arrow] (dec1) -- node[anchor=south] {a-bca-cb} (out1);
\draw [arrow] (out1) -- node[anchor=east] {$\alpha_0 \cdot A_a \cdot A_{bca} \cdot A_{cb} \cdot \alpha_{\infty}$} (stop);

\end{tikzpicture}
\end{frame}
\end{comment}

%------------------------------------------------

\begin{frame}
\frametitle{The Big Picture}

\begin{columns}[c]

\column{1\textwidth} % Left column and width
\begin{figure}

\includegraphics[width=0.63\linewidth]{uCOREPICS/M-PSR_Flow_Chart.pdf}
\end{figure}
\end{columns}

\end{frame}

%------------------------------------------------

\begin{frame}
\Huge{\centerline{Questions?}}
\end{frame}




%					DONE

%------------------------------------------------
%------------------------------------------------
%------------------------------------------------
%------------------------------------------------
%-----------------------------------------------
%------------------------------------------------
%------------------------------------------------

%					DONE


%----------------------------------------------------------------------------------------

\end{document} 